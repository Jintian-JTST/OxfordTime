\documentclass[10pt,a4paper]{article}
\usepackage[utf8]{inputenc}
\usepackage[ngerman]{babel}
\usepackage[T1]{fontenc}
\usepackage{amsmath}
\usepackage{amsfonts}
\usepackage{amssymb}
\usepackage{graphicx}
\usepackage{lmodern}
\usepackage{physics}
\usepackage[left=2cm,right=2cm,top=2cm,bottom=2cm]{geometry}
\usepackage{siunitx}
\usepackage{fancyhdr}
\usepackage{enumerate}
\usepackage{mhchem}
\usepackage{mathtools}
\usepackage{float}
\usepackage{xcolor}
\usepackage{mdframed}
\usepackage{csquotes}
\usepackage{extarrows}

\addto\captionsngerman{%
  \renewcommand{\contentsname}{Contents}
}

% Cover Information
\def\univname{University of Oxford}
\def\coursenum{PHYSICS\ UNDERGRADUATE}
\def\coursename{Problem Sheet Answers}
\def\professor{Instructor Name}
\def\student{Jintian Wang}
\def\semesteryear{Semester Year}
\def\imagename{beltcrest.pdf}		    % Replace with University Seal
\def\scalesize{0.25}					% Scale Logo Size 
% Title Page


\definecolor{UnivColor}{RGB}{0,33,71}   % 牛津深蓝随便取个
% 或
\definecolor{UnivColor}{HTML}{002147}   % 更像牛津蓝
\newcommand{\e}{\mathrm{e}}


\newcommand{\dx}{\frac{\mathrm{d}}{\mathrm{d}x}}
\newcommand{\dy}{\frac{\mathrm{d}}{\mathrm{d}y}}
\newcommand{\dydx}{\frac{\mathrm{dy}}{\mathrm{d}x}}

\newcommand{\coverpage}{%
\pagestyle{empty}
\begin{center}
\vspace*{5cm}

\includegraphics[scale=1.5]{\imagename}

\vspace{1cm}
{\color{UnivColor}\sffamily\bfseries\Huge \univname}
\end{center}

\vspace{2in}
{\color{UnivColor}\sffamily\bfseries\Huge\noindent 
{\large\coursenum} \par\vspace{0.2cm}\noindent\coursename\par\vspace{-0.5cm}\noindent\rule{0.65\textwidth}{0.05cm}\par\vspace{0.2cm}}
{\color{UnivColor}\sffamily\bfseries \Large\noindent Notes By: \student}
\pagebreak
}



% Siunitx settings
\sisetup{locale=DE}
\sisetup{per-mode = symbol-or-fraction}
\DeclareSIUnit\year{a}
\DeclareSIUnit\clight{c}

% Box style
\mdfdefinestyle{exercise}{
    backgroundcolor=black!10,roundcorner=8pt,hidealllines=true,nobreak
}

\newcommand{\se}{\subseteq}

\begin{document}

\lhead{Mathematics Methods}
\rhead{Collection}

\begin{enumerate}
    \setcounter{enumi}{0}
    \item \textbf{Complex Numbers} [7]
    \begin{mdframed}[style=exercise]
        \begin{enumerate}
        \item[(a)] Express the term $(1+i)^4$ in the form $r\e^{i\theta}$, where $r$ and $\theta$ are real variables.
        \item[(b)] Express the complex number $\tan^{-1}(2i)$ in the form $x+iy$ where $x,y$ are real.
        \item[(c)] Given that $z=z_1+z_2$, where $z_1=\e^{i\theta_1}$, $z_2=\e^{i\theta_2}$ and $\theta_1,\theta_2$ are real variables, find an expression for $|z|$ in terms of $\Delta\theta=\theta_1-\theta_2$.
        \end{enumerate}
    \end{mdframed}

    \begin{enumerate}
        \item [(a)] \begin{align*}
           \displaystyle (1+i)^4&=\left(\sqrt{2}\e^{i\pi/4}\right)^4=\boxed{4\e^{i\pi}}
        \end{align*}
        \item [(b)] Let $w=\tan^{-1}(2i)$, then $\tan(w)=2i$. Using the identity $\displaystyle \tan(w)=\frac{\sin(w)}{\cos(w)}$, we have
        \begin{align*}
            \frac{\e^{iw}-\e^{-iw}}{i(\e^{iw}+\e^{-iw})}&=2i\\
                        \e^{iw}-\e^{-iw}&=-2i(\e^{iw}+\e^{-iw})\\
                        3\e^{iw}+\e^{-iw}&=0\\
                        \e^{2iw}&=-\frac{1}{3}\\
                        2iw&=\ln\left(-\frac{1}{3}\right)\\
                        w&=-\frac{i}{2}\left(\ln\frac{1}{3}+i\pi\right)=\boxed{\frac{\pi}{2}+\frac{i}{2}\ln 3}
        \end{align*}
        \item [(c)] \begin{align*}
            |z|^2&=z\bar{z}=(z_1+z_2)(\bar{z_1}+\bar{z_2})\\
            &=|z_1|^2+|z_2|^2+z_1\bar{z_2}+\bar{z_1}z_2\\
            &=2+\e^{i(\theta_1-\theta_2)}+\e^{-i(\theta_1-\theta_2)}\\
            &=2+2\cos(\Delta\theta)\\
            \Rightarrow |z|&=\boxed{\sqrt{2\left(1+\cos(\Delta\theta)\right)}}
        \end{align*}
    \end{enumerate}


    \item \textbf{Vectors} [8]
    \begin{mdframed}[style=exercise]
        \begin{enumerate}
        \item[(a)] Write down the equation of the plane
        \[
        3x+4y+5z=10
        \]
        in the vector form
        \[
        \mathbf{r}=\mathbf{r}_0+t_1\mathbf{a}+t_2\mathbf{b},
        \]
        where $\mathbf{a}$ and $\mathbf{b}$ are constant vectors in the plane, $t_1$ and $t_2$ are real parameters and $\mathbf{r}_0$ is a constant vector.
        What is the distance between the plane and the origin?
        \item[(b)] Let $\mathbf{a},\mathbf{b},\mathbf{c}$ be unit vectors. Show that
        \[
        (\mathbf{a}\times\mathbf{b})\cdot(\mathbf{a}\times\mathbf{c})
        = \mathbf{b}\cdot\mathbf{c} - (\mathbf{a}\cdot\mathbf{b})(\mathbf{a}\cdot\mathbf{c}).
        \]
        \end{enumerate}
    \end{mdframed}

    \begin{enumerate}
        \item [(a)] A particular solution to the plane equation is $\mathbf{r_0}=\left(0,0,2\right)^T$. Two independent vectors in the plane are $\mathbf{a}=(0,-2.5,2)^T$ and $\mathbf{b}=(-10/3,0,2)^T$. Thus the vector form of the plane is
        \[\mathbf{r}=\begin{pmatrix}0\\ 0\\ 2\end{pmatrix}+t_1\begin{pmatrix}0\\ -2.5\\ 2\end{pmatrix}+t_2\begin{pmatrix}-10/3\\ 0\\ 2\end{pmatrix}. \]
        The distance from the origin to the plane is given by
        \[ d = \frac{|\mathbf{r_0} \cdot \mathbf{n}|}{|\mathbf{n}|}, \]
        where $\mathbf{n}=(3,4,5)$ is the normal vector to the plane. Substituting $\mathbf{r_0}=\left(0,0,2\right)$ and $\mathbf{n}=(3,4,5)$,
        \[ d = \frac{\left|\left(0,0,2\right) \cdot (3,4,5)\right|}{\sqrt{3^2+4^2+5^2}} = \frac{\left|10\right|}{\sqrt{50}} = \frac{10}{\sqrt{50}} = \boxed{\sqrt{2}}.\]


        \item [(b)] Using the vector triple product identity $\mathbf{x}\cdot(\mathbf{y}\times\mathbf{z}) = \mathbf{y}\cdot(\mathbf{z}\times\mathbf{x}) = \mathbf{z}\cdot(\mathbf{x}\times\mathbf{y})$, let $\mathbf{x}=\mathbf{a}$, $\mathbf{y}=\mathbf{b}$ and $\mathbf{z}=\mathbf{a}\times\mathbf{c}$, we have
        \begin{align*}
            (\mathbf{a}\times\mathbf{b})\cdot(\mathbf{a}\times\mathbf{c})&= \mathbf{a}\cdot\left[\mathbf{b}\times(\mathbf{a}\times\mathbf{c})\right]
        \end{align*}
        Since $(\textbf{b}\times(\textbf{a}\times\textbf{c}))_i=\epsilon_{ijk}b_j(a\times c)_k=\epsilon_{ijk}b_j\epsilon_{klm}a_kc_m=\epsilon_{ijk}\epsilon_{klm}b_ja_kc_m$, and using the identity $\epsilon_{ijk}\epsilon_{klm}=\delta_{il}\delta_{jm}-\delta_{im}\delta_{jl}$, we get $\epsilon_{ijk}\epsilon_{klm}b_ja_kc_m=(\delta_{il}\delta_{jm}-\delta_{im}\delta_{jl})b_ja_kc_m= b_ja_ic_m\delta_{il}\delta_{jm}-b_ja_kc_m\delta_{im}\delta_{jl}=a_i(b_jc_j)-c_i(a_jb_j)$, which is just $\mathbf{a}(\mathbf{b}\cdot\mathbf{c})-\mathbf{c}(\mathbf{a}\cdot\mathbf{b})$. Continuing from above,
        \begin{align*}
            (\mathbf{a}\times\mathbf{b})\cdot(\mathbf{a}\times\mathbf{c})&= \mathbf{a}\cdot\left[\mathbf{a}(\mathbf{b}\cdot\mathbf{c}) - \mathbf{c}(\mathbf{a}\cdot\mathbf{b})\right]\\
            &= (\mathbf{a}\cdot\mathbf{a})(\mathbf{b}\cdot\mathbf{c}) - (\mathbf{a}\cdot\mathbf{c})(\mathbf{a}\cdot\mathbf{b})\\
            &= 1\cdot(\mathbf{b}\cdot\mathbf{c}) - (\mathbf{a}\cdot\mathbf{b})(\mathbf{a}\cdot\mathbf{c})\\
            &= {\mathbf{b}\cdot\mathbf{c} - (\mathbf{a}\cdot\mathbf{b})(\mathbf{a}\cdot\mathbf{c})}   
        \end{align*}
        \hfill Q.E.D.
    \end{enumerate}



    \item \textbf{Matrix and linear equation} [5]
    \begin{mdframed}[style=exercise]
        Consider the set of linear equations
        \[
        \begin{aligned}
        2x+y+z &= 2b,\\
        ax+3y+2z &= 2a,\\
        2x+y+3z &= 4,
        \end{aligned}
        \]
        where $x,y,z$ are real variables and $a,b$ are real parameters.

        Find the values of $a$ and $b$ for which the set of equations have:
        \begin{enumerate}
        \item[(i)] a unique solution,
        \item[(ii)] infinitely many solutions,
        \item[(iii)] no solution.
        \end{enumerate}

    \end{mdframed}


\[
\begin{aligned}
2x+y+z &= 2b,\\
ax+3y+2z &= 2a,\\
2x+y+3z &= 4.
\end{aligned}
\]

Let the coefficient matrix be
\[
A=
\begin{pmatrix}
2 & 1 & 1\\
a & 3 & 2\\
2 & 1 & 3
\end{pmatrix},
\qquad
\mathbf{c}=
\begin{pmatrix}
2b\\
2a\\
4
\end{pmatrix}.
\]
The determinant of $A$ is
\[
\det A
=
\begin{vmatrix}
2 & 1 & 1\\
a & 3 & 2\\
2 & 1 & 3
\end{vmatrix}
=-2(a-6).
\]

\begin{enumerate}
\item[(i)] \textbf{Unique solution.}  
If $\det A \neq 0$, i.e.\ $a \neq 6$, the system has a unique solution for all values of $b$.

\item[(ii)] \textbf{Infinitely many solutions.}  
Let $a=6$. The system becomes
\[
\begin{aligned}
2x+y+z &= 2b,\\
6x+3y+2z &= 12,\\
2x+y+3z &= 4.
\end{aligned}
\]
Subtracting the first equation from the third gives
\[
2z = 4 - 2b \quad \Rightarrow \quad z = 2 - b.
\]
Dividing the second equation by $3$ yields
\[
2x+y+\frac{2}{3}z = 4.
\]
From the first equation, $2x+y = 2b - z$. Substituting,
\[
2b - z + \frac{2}{3}z = 4
\quad \Rightarrow \quad
2b - \frac{1}{3}z = 4
\quad \Rightarrow \quad
z = 6b - 12.
\]
Consistency requires
\[
2 - b = 6b - 12 \quad \Rightarrow \quad b = 2.
\]
Hence $z=0$ and the remaining equation is
\[
2x+y = 4.
\]
Letting $x=t$, the solutions are
\[
(x,y,z) = (t,\,4-2t,\,0), \qquad t\in\mathbb{R},
\]
so there are infinitely many solutions when $(a,b)=(6,2)$.

\item[(iii)] \textbf{No solution.}  
If $a=6$ and $b\neq 2$, the two expressions for $z$ are inconsistent. Hence the system has no solution.
\end{enumerate}

\[
\boxed{
\begin{array}{ll}
\text{Unique solution} & a\neq 6 \ (\text{any } b),\\[4pt]
\text{Infinitely many solutions} & a=6,\ b=2,\\[4pt]
\text{No solution} & a=6,\ b\neq 2.
\end{array}}
\]



    
    \item \textbf{Differential equation} [5]
    \begin{mdframed}[style=exercise]
        Find the general solution for the differential equation
\[
\dx\left(x^2\dydx\right)=n(n+1)y,
\]
where $x$ is a real variable and $n$ is a real constant.
    \end{mdframed}

    Using the product rule, we have
    \[\dx\left(x^2\dydx\right) = 2x\dydx + x^2\frac{\mathrm{d}^2y}{\mathrm{d}x^2}.\] Thus the differential equation can be rewritten as
    \[ x^2\frac{\mathrm{d}^2y}{\mathrm{d}x^2} + 2x\dydx - n(n+1)y = 0.\]
    This is an Euler-Cauchy equation. We try a solution of the form $y=x^m$, where $m$ is a constant to be determined. Substituting this into the differential equation, we get
    \[ x^2\cdot m(m-1)x^{m-2} + 2x\cdot mx^{m-1} - n(n+1)x^m = 0.\]
    Simplifying, we have  $x^m\left[m(m-1) + 2m - n(n+1)\right] = 0$, which gives us the characteristic equation
    \[ m^2 + m - n(n+1) = 0.\]
    Solving for $m$, we get $m = \frac{-1 \pm (2n+1)}{2}$, so $m_1 = n$ and $m_2 = -(n+1)$.

    The general solution is then
    \[ y(x) = A x^n + B x^{-(n+1)},\]
    where $A$ and $B$ are arbitrary constants.

    \item \textbf{Matrix and properties} [7]
    \begin{mdframed}[style=exercise]
Let $A$ and $B$ be $n\times n$ Hermitian matrices and $U$ an $n\times n$ unitary matrix.
\begin{enumerate}
\item[(a)] Show that the modulus of each of the eigenvalues of $U$ is equal to one $(|\lambda|=1)$.
\item[(b)] Show that the eigenvalues of $A$ are real.
\item[(c)] Assuming that $U=A+iB$, show that
\begin{enumerate}
\item[(i)] $A^2+B^2=I$, where $I$ is the identity matrix,
\item[(ii)] $AB-BA=0$.
\end{enumerate}
\end{enumerate}
    \end{mdframed}

    \begin{enumerate}
        \item [(a)] Let $\lambda$ be an eigenvalue of $U$ with corresponding eigenvector $\mathbf{v}$, so that $U\mathbf{v}=\lambda \mathbf{v}$. Taking the norm on both sides, we have
        \[ \|U\mathbf{v}\| = \|\lambda \mathbf{v}\|.\]
        Since $U$ is unitary, it preserves the norm, so $\|U\mathbf{v}\| = \|\mathbf{v}\|$. Also, $\|\lambda \mathbf{v}\| = |\lambda| \|\mathbf{v}\|$. Therefore,
        \[ \|\mathbf{v}\| = |\lambda| \|\mathbf{v}\|.\]
        Since $\mathbf{v}$ is a non-zero eigenvector, $\|\mathbf{v}\| \neq 0$, we can divide both sides by $\|\mathbf{v}\|$ to get
        \[ 1 = |\lambda|.\]
        Thus, the modulus of each eigenvalue of $U$ is equal to one.

        \item [(b)] Let $\mu$ be an eigenvalue of $A$ with corresponding eigenvector $\mathbf{w}$, so that $A\mathbf{w}=\mu \mathbf{w}$. Taking the conjugate transpose of both sides, we have
        \[ \mathbf{w}^\dagger A^\dagger = \mu^* \mathbf{w}^\dagger.\]
        Since $A$ is Hermitian, $A^\dagger = A$. Thus,
        \[ \mathbf{w}^\dagger A = \mu^* \mathbf{w}^\dagger.\]
        Multiplying both sides by $\mathbf{w}$ from the right, we get
        \[ \mathbf{w}^\dagger A\mathbf{w} = \mu^* \mathbf{w}^\dagger\mathbf{w}.\]
        On the other hand, from the original eigenvalue equation,
        \[ \mathbf{w}^\dagger A\mathbf{w} = \mu \mathbf{w}^\dagger\mathbf{w}.\]
        Equating the two expressions for $\mathbf{w}^\dagger A\mathbf{w}$, we have 
        \[ \mu \mathbf{w}^\dagger\mathbf{w} = \mu^* \mathbf{w}^\dagger\mathbf{w}.\]
        Since $\mathbf{w}$ is a non-zero eigenvector, $\mathbf{w}^\dagger\mathbf{w} \neq 0$, we can divide both sides by $\mathbf{w}^\dagger\mathbf{w}$ to get
        \[ \mu = \mu^*.\]
        Thus, the eigenvalues of $A$ are real.


        \item [(c)] Given that $U=A+iB$ is unitary, we have
        \[ U^\dagger U = I.\]
        Calculating $U^\dagger U$, we get
        \[ (A - iB)(A + iB) = A^2 + iAB - iBA + B^2 = A^2 + B^2 + i(AB - BA).\]
        Setting this equal to the identity matrix $I$, we have
        \[ A^2 + B^2 + i(AB - BA) = I.\]
        Equating the real and imaginary parts, we obtain the two equations:
        \begin{enumerate}
            \item[(i)] Real part: 
            \[ A^2 + B^2 = I.\]
            \item[(ii)] Imaginary part: 
            \[ AB - BA = 0.\]
        \end{enumerate}
        \end{enumerate}





    \item 
\end{enumerate}





\end{document}