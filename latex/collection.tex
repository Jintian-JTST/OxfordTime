\documentclass[10pt,a4paper]{article}
\usepackage[utf8]{inputenc}
\usepackage[english]{babel}
\usepackage[T1]{fontenc}
\usepackage{amsmath}
\usepackage{amsfonts}
\usepackage{amssymb}
\usepackage{graphicx}
\usepackage{lmodern}
\usepackage{physics}
\usepackage[left=2cm,right=2cm,top=2cm,bottom=2cm]{geometry}
\usepackage{siunitx}
\usepackage{fancyhdr}
\usepackage{enumerate}
\usepackage{mhchem}
\usepackage{mathtools}
\usepackage{float}
\usepackage{xcolor}
\usepackage{mdframed}
\usepackage{csquotes}
\usepackage{extarrows}

\addto\captionsngerman{%
  \renewcommand{\contentsname}{Contents}
}

% Cover Information
\def\univname{University of Oxford}
\def\coursenum{PHYSICS\ UNDERGRADUATE}
\def\coursename{Problem Sheet Answers}
\def\professor{Instructor Name}
\def\student{Jintian Wang}
\def\semesteryear{Semester Year}
\def\imagename{beltcrest.pdf}		    % Replace with University Seal
\def\scalesize{0.25}					% Scale Logo Size 
% Title Page


\definecolor{UnivColor}{RGB}{0,33,71}   % 牛津深蓝随便取个
% 或
\definecolor{UnivColor}{HTML}{002147}   % 更像牛津蓝
\newcommand{\e}{\mathrm{e}}


\newcommand{\dx}{\frac{\mathrm{d}}{\mathrm{d}x}}
\newcommand{\dy}{\frac{\mathrm{d}}{\mathrm{d}y}}
\newcommand{\dydx}{\frac{\mathrm{dy}}{\mathrm{d}x}}

\newcommand{\coverpage}{%
\pagestyle{empty}
\begin{center}
\vspace*{5cm}

\includegraphics[scale=1.5]{\imagename}

\vspace{1cm}
{\color{UnivColor}\sffamily\bfseries\Huge \univname}
\end{center}

\vspace{2in}
{\color{UnivColor}\sffamily\bfseries\Huge\noindent 
{\large\coursenum} \par\vspace{0.2cm}\noindent\coursename\par\vspace{-0.5cm}\noindent\rule{0.65\textwidth}{0.05cm}\par\vspace{0.2cm}}
{\color{UnivColor}\sffamily\bfseries \Large\noindent Notes By: \student}
\pagebreak
}



% Siunitx settings
\sisetup{locale=DE}
\sisetup{per-mode = symbol-or-fraction}
\DeclareSIUnit\year{a}
\DeclareSIUnit\clight{c}

% Box style
\mdfdefinestyle{exercise}{
    backgroundcolor=black!10,roundcorner=8pt,hidealllines=true,nobreak
}

\title{\textbf{Answers for Collection of Mathematics Methods}}
\author{JTST}
\date{\today}
\newcommand{\se}{\subseteq}

\begin{document}
\maketitle
\lhead{Mathematics Methods}
\rhead{Collection}
\section*{Section A}
\begin{enumerate}
    \setcounter{enumi}{0}
    \item \textbf{Complex Numbers} [7]
    \begin{mdframed}[style=exercise]
        \begin{enumerate}
        \item[(a)] Express the term $(1+i)^4$ in the form $r\e^{i\theta}$, where $r$ and $\theta$ are real variables.
        \item[(b)] Express the complex number $\tan^{-1}(2i)$ in the form $x+iy$ where $x,y$ are real.
        \item[(c)] Given that $z=z_1+z_2$, where $z_1=\e^{i\theta_1}$, $z_2=\e^{i\theta_2}$ and $\theta_1,\theta_2$ are real variables, find an expression for $|z|$ in terms of $\Delta\theta=\theta_1-\theta_2$.
        \end{enumerate}
    \end{mdframed}

    \begin{enumerate}
        \item [(a)] \begin{align*}
           \displaystyle (1+i)^4&=\left(\sqrt{2}\e^{i\pi/4}\right)^4=\boxed{4\e^{i\pi}}
        \end{align*}
        \item [(b)] Let $w=\tan^{-1}(2i)$, then $\tan(w)=2i$. Using the identity $\displaystyle \tan(w)=\frac{\sin(w)}{\cos(w)}$, we have
        \begin{align*}
            \frac{\e^{iw}-\e^{-iw}}{i(\e^{iw}+\e^{-iw})}&=2i\\
                        \e^{iw}-\e^{-iw}&=-2i(\e^{iw}+\e^{-iw})\\
                        3\e^{iw}+\e^{-iw}&=0\\
                        \e^{2iw}&=-\frac{1}{3}\\
                        2iw&=\ln\left(-\frac{1}{3}\right)\\
                        w&=-\frac{i}{2}\left(\ln\frac{1}{3}+i\pi\right)=\boxed{\frac{\pi}{2}+\frac{i}{2}\ln 3}
        \end{align*}
        \item [(c)] \begin{align*}
            |z|^2&=z\bar{z}=(z_1+z_2)(\bar{z_1}+\bar{z_2})\\
            &=|z_1|^2+|z_2|^2+z_1\bar{z_2}+\bar{z_1}z_2\\
            &=2+\e^{i(\theta_1-\theta_2)}+\e^{-i(\theta_1-\theta_2)}\\
            &=2+2\cos(\Delta\theta)\\
            \Rightarrow |z|&=\boxed{\sqrt{2\left(1+\cos(\Delta\theta)\right)}}
        \end{align*}
    \end{enumerate}


    \item \textbf{Vectors} [8]
    \begin{mdframed}[style=exercise]
        \begin{enumerate}
        \item[(a)] Write down the equation of the plane
        \[
        3x+4y+5z=10
        \]
        in the vector form
        \[
        \mathbf{r}=\mathbf{r}_0+t_1\mathbf{a}+t_2\mathbf{b},
        \]
        where $\mathbf{a}$ and $\mathbf{b}$ are constant vectors in the plane, $t_1$ and $t_2$ are real parameters and $\mathbf{r}_0$ is a constant vector.
        What is the distance between the plane and the origin?
        \item[(b)] Let $\mathbf{a},\mathbf{b},\mathbf{c}$ be unit vectors. Show that
        \[
        (\mathbf{a}\times\mathbf{b})\cdot(\mathbf{a}\times\mathbf{c})
        = \mathbf{b}\cdot\mathbf{c} - (\mathbf{a}\cdot\mathbf{b})(\mathbf{a}\cdot\mathbf{c}).
        \]
        \end{enumerate}
    \end{mdframed}

    \begin{enumerate}
        \item [(a)] A particular solution to the plane equation is $\mathbf{r_0}=\left(0,0,2\right)^T$. Two independent vectors in the plane are $\mathbf{a}=(0,-2.5,2)^T$ and $\mathbf{b}=(-10/3,0,2)^T$. Thus the vector form of the plane is
        \[\mathbf{r}=\begin{pmatrix}0\\ 0\\ 2\end{pmatrix}+t_1\begin{pmatrix}0\\ -2.5\\ 2\end{pmatrix}+t_2\begin{pmatrix}-10/3\\ 0\\ 2\end{pmatrix}. \]
        The distance from the origin to the plane is given by
        \[ d = \frac{|\mathbf{r_0} \cdot \mathbf{n}|}{|\mathbf{n}|}, \]
        where $\mathbf{n}=(3,4,5)$ is the normal vector to the plane. Substituting $\mathbf{r_0}=\left(0,0,2\right)$ and $\mathbf{n}=(3,4,5)$,
        \[ d = \frac{\left|\left(0,0,2\right) \cdot (3,4,5)\right|}{\sqrt{3^2+4^2+5^2}} = \frac{\left|10\right|}{\sqrt{50}} = \frac{10}{\sqrt{50}} = \boxed{\sqrt{2}}.\]


        \item [(b)] Using the vector triple product identity $\mathbf{x}\cdot(\mathbf{y}\times\mathbf{z}) = \mathbf{y}\cdot(\mathbf{z}\times\mathbf{x}) = \mathbf{z}\cdot(\mathbf{x}\times\mathbf{y})$, let $\mathbf{x}=\mathbf{a}$, $\mathbf{y}=\mathbf{b}$ and $\mathbf{z}=\mathbf{a}\times\mathbf{c}$, we have
        \begin{align*}
            (\mathbf{a}\times\mathbf{b})\cdot(\mathbf{a}\times\mathbf{c})&= \mathbf{a}\cdot\left[\mathbf{b}\times(\mathbf{a}\times\mathbf{c})\right]
        \end{align*}
        Since $(\textbf{b}\times(\textbf{a}\times\textbf{c}))_i=\epsilon_{ijk}b_j(a\times c)_k=\epsilon_{ijk}b_j\epsilon_{klm}a_kc_m=\epsilon_{ijk}\epsilon_{klm}b_ja_kc_m$, and using the identity $\epsilon_{ijk}\epsilon_{klm}=\delta_{il}\delta_{jm}-\delta_{im}\delta_{jl}$, we get $\epsilon_{ijk}\epsilon_{klm}b_ja_kc_m=(\delta_{il}\delta_{jm}-\delta_{im}\delta_{jl})b_ja_kc_m= b_ja_ic_m\delta_{il}\delta_{jm}-b_ja_kc_m\delta_{im}\delta_{jl}=a_i(b_jc_j)-c_i(a_jb_j)$, which is just $\mathbf{a}(\mathbf{b}\cdot\mathbf{c})-\mathbf{c}(\mathbf{a}\cdot\mathbf{b})$. Continuing from above,
        \begin{align*}
            (\mathbf{a}\times\mathbf{b})\cdot(\mathbf{a}\times\mathbf{c})&= \mathbf{a}\cdot\left[\mathbf{a}(\mathbf{b}\cdot\mathbf{c}) - \mathbf{c}(\mathbf{a}\cdot\mathbf{b})\right]\\
            &= (\mathbf{a}\cdot\mathbf{a})(\mathbf{b}\cdot\mathbf{c}) - (\mathbf{a}\cdot\mathbf{c})(\mathbf{a}\cdot\mathbf{b})\\
            &= 1\cdot(\mathbf{b}\cdot\mathbf{c}) - (\mathbf{a}\cdot\mathbf{b})(\mathbf{a}\cdot\mathbf{c})\\
            &= {\mathbf{b}\cdot\mathbf{c} - (\mathbf{a}\cdot\mathbf{b})(\mathbf{a}\cdot\mathbf{c})}   
        \end{align*}
        \hfill Q.E.D.
    \end{enumerate}



    \item \textbf{Matrix and linear equation} [5]
    \begin{mdframed}[style=exercise]
        Consider the set of linear equations
        \[
        \begin{aligned}
        2x+y+z &= 2b,\\
        ax+3y+2z &= 2a,\\
        2x+y+3z &= 4,
        \end{aligned}
        \]
        where $x,y,z$ are real variables and $a,b$ are real parameters.

        Find the values of $a$ and $b$ for which the set of equations have:
        \begin{enumerate}
        \item[(i)] a unique solution,
        \item[(ii)] infinitely many solutions,
        \item[(iii)] no solution.
        \end{enumerate}

    \end{mdframed}


\[
\begin{aligned}
2x+y+z &= 2b,\\
ax+3y+2z &= 2a,\\
2x+y+3z &= 4.
\end{aligned}
\]

Let the coefficient matrix be
\[
A=
\begin{pmatrix}
2 & 1 & 1\\
a & 3 & 2\\
2 & 1 & 3
\end{pmatrix},
\qquad
\mathbf{c}=
\begin{pmatrix}
2b\\
2a\\
4
\end{pmatrix}.
\]
The determinant of $A$ is
\[
\det A
=
\begin{vmatrix}
2 & 1 & 1\\
a & 3 & 2\\
2 & 1 & 3
\end{vmatrix}
=-2(a-6).
\]

\begin{enumerate}
\item[(i)] \textbf{Unique solution.}  
If $\det A \neq 0$, i.e.\ $a \neq 6$, the system has a unique solution for all values of $b$.

\item[(ii)] \textbf{Infinitely many solutions.}  
Let $a=6$. The system becomes
\[
\begin{aligned}
2x+y+z &= 2b,\\
6x+3y+2z &= 12,\\
2x+y+3z &= 4.
\end{aligned}
\]
Subtracting the first equation from the third gives
\[
2z = 4 - 2b \quad \Rightarrow \quad z = 2 - b.
\]
Dividing the second equation by $3$ yields
\[
2x+y+\frac{2}{3}z = 4.
\]
From the first equation, $2x+y = 2b - z$. Substituting,
\[
2b - z + \frac{2}{3}z = 4
\quad \Rightarrow \quad
2b - \frac{1}{3}z = 4
\quad \Rightarrow \quad
z = 6b - 12.
\]
Consistency requires
\[
2 - b = 6b - 12 \quad \Rightarrow \quad b = 2.
\]
Hence $z=0$ and the remaining equation is
\[
2x+y = 4.
\]
Letting $x=t$, the solutions are
\[
(x,y,z) = (t,\,4-2t,\,0), \qquad t\in\mathbb{R},
\]
so there are infinitely many solutions when $(a,b)=(6,2)$.

\item[(iii)] \textbf{No solution.}  
If $a=6$ and $b\neq 2$, the two expressions for $z$ are inconsistent. Hence the system has no solution.
\end{enumerate}

\[
\boxed{
\begin{array}{ll}
\text{Unique solution} & a\neq 6 \ (\text{any } b),\\[4pt]
\text{Infinitely many solutions} & a=6,\ b=2,\\[4pt]
\text{No solution} & a=6,\ b\neq 2.
\end{array}}
\]



    
    \item \textbf{Differential equation} [5]
    \begin{mdframed}[style=exercise]
        Find the general solution for the differential equation
\[
\dx\left(x^2\dydx\right)=n(n+1)y,
\]
where $x$ is a real variable and $n$ is a real constant.
    \end{mdframed}

    Using the product rule, we have
    \[\dx\left(x^2\dydx\right) = 2x\dydx + x^2\frac{\mathrm{d}^2y}{\mathrm{d}x^2}.\] Thus the differential equation can be rewritten as
    \[ x^2\frac{\mathrm{d}^2y}{\mathrm{d}x^2} + 2x\dydx - n(n+1)y = 0.\]
    This is an Euler-Cauchy equation. We try a solution of the form $y=x^m$, where $m$ is a constant to be determined. Substituting this into the differential equation, we get
    \[ x^2\cdot m(m-1)x^{m-2} + 2x\cdot mx^{m-1} - n(n+1)x^m = 0.\]
    Simplifying, we have  $x^m\left[m(m-1) + 2m - n(n+1)\right] = 0$, which gives us the characteristic equation
    \[ m^2 + m - n(n+1) = 0.\]
    Solving for $m$, we get $m = \frac{-1 \pm (2n+1)}{2}$, so $m_1 = n$ and $m_2 = -(n+1)$.

    The general solution is then
    \[ y(x) = A x^n + B x^{-(n+1)},\]
    where $A$ and $B$ are arbitrary constants.

    \newpage
    \item \textbf{Matrix and properties} [7]
    \begin{mdframed}[style=exercise]
Let $A$ and $B$ be $n\times n$ Hermitian matrices and $U$ an $n\times n$ unitary matrix.
\begin{enumerate}
\item[(a)] Show that the modulus of each of the eigenvalues of $U$ is equal to one $(|\lambda|=1)$.
\item[(b)] Show that the eigenvalues of $A$ are real.
\item[(c)] Assuming that $U=A+iB$, show that
\begin{enumerate}
\item[(i)] $A^2+B^2=I$, where $I$ is the identity matrix,
\item[(ii)] $AB-BA=0$.
\end{enumerate}
\end{enumerate}
    \end{mdframed}

    \begin{enumerate}
        \item [(a)] Let $\lambda$ be an eigenvalue of $U$ with corresponding eigenvector $\mathbf{v}$, so that $U\mathbf{v}=\lambda \mathbf{v}$. Taking the conjugate transpose of both sides, we have
        \[ \mathbf{v}^\dagger U^\dagger = \lambda^* \mathbf{v}^\dagger.\]
        \[ \mathbf{v}^\dagger U^\dagger U=\mathbf{v}^\dagger = \lambda^* \mathbf{v}^\dagger U.\]
        Multiplying both sides by $U\mathbf{v}$ from the right, we get
        \[ \mathbf{v}^\dagger U\mathbf{v} = \lambda^* \mathbf{v}^\dagger U U\mathbf{v} = \lambda^* \mathbf{v}^\dagger \mathbf{v}.\]
        On the other hand, from the original eigenvalue equation,
        \[ \mathbf{v}^\dagger U\mathbf{v} = \lambda \mathbf{v}^\dagger\mathbf{v}.\]
        Equating the two expressions for $\mathbf{v}^\dagger U\mathbf{v}$, we have 
        \[ \lambda \mathbf{v}^\dagger\mathbf{v} = \lambda^* \mathbf{v}^\dagger\mathbf{v}.\]
        Since $\mathbf{v}$ is a non-zero eigenvector, $\mathbf{v}^\dagger\mathbf{v} \neq 0$, we can divide both sides by $\mathbf{v}^\dagger\mathbf{v}$ to get
        \[ \lambda = \lambda^*.\]
        Thus, we have
        \[ |\lambda|^2 = \lambda \lambda^* = 1.\]
        \hfill Q.E.D.

        \item [(b)] Let $\mu$ be an eigenvalue of $A$ with corresponding eigenvector $\mathbf{w}$, so that $A\mathbf{w}=\mu \mathbf{w}$. Taking the conjugate transpose of both sides, we have
        \[ \mathbf{w}^\dagger A^\dagger = \mu^* \mathbf{w}^\dagger.\]
        Since $A$ is Hermitian, $A^\dagger = A$. Thus,
        \[ \mathbf{w}^\dagger A = \mu^* \mathbf{w}^\dagger.\]
        Multiplying both sides by $\mathbf{w}$ from the right, we get
        \[ \mathbf{w}^\dagger A\mathbf{w} = \mu^* \mathbf{w}^\dagger\mathbf{w}.\]
        On the other hand, from the original eigenvalue equation,
        \[ \mathbf{w}^\dagger A\mathbf{w} = \mu \mathbf{w}^\dagger\mathbf{w}.\]
        Equating the two expressions for $\mathbf{w}^\dagger A\mathbf{w}$, we have 
        \[ \mu \mathbf{w}^\dagger\mathbf{w} = \mu^* \mathbf{w}^\dagger\mathbf{w}.\]
        Since $\mathbf{w}$ is a non-zero eigenvector, $\mathbf{w}^\dagger\mathbf{w} \neq 0$, we can divide both sides by $\mathbf{w}^\dagger\mathbf{w}$ to get
        \[ \mu = \mu^*.\]
        \hfill Q.E.D.


        \item [(c)] Given that $U=A+iB$ is unitary, we have
        \[ U^\dagger U = I.\]
        Calculating $U^\dagger U$, we get
        \[ (A - iB)(A + iB) = A^2 + iAB - iBA + B^2 = A^2 + B^2 + i(AB - BA).\]
        Setting this equal to the identity matrix $I$, we have
        \[ A^2 + B^2 + i(AB - BA) = I.\]
        \hfill Q.E.D.
        \end{enumerate}




    \newpage
    \item \textbf{Matrix and geometry} [8]
    \begin{mdframed}[style=exercise]
        The rotation matrix $A$ in $\mathbb{R}^3$ is given by
        \[
        A=\frac12
        \begin{pmatrix}
        \sqrt2 & -1 & -1\\
        0 & \sqrt2 & -\sqrt2\\
        \sqrt2 & 1 & 1
        \end{pmatrix}.
        \]
        \begin{enumerate}
        \item[(a)] Show that the matrix $A$ is orthogonal.
        \item[(b)] Calculate $\cos\theta$, where $\theta$ is the angle of rotation, and find a unit vector in the direction of the axis of rotation.
        \end{enumerate}
    \end{mdframed}

    \begin{enumerate}
        \item [(a)] To show that $A$ is orthogonal, we need to verify that $A^TA=I$. Calculating $A^T$,
        \[
        A^T=\frac12
        \begin{pmatrix}
        \sqrt2 & 0 & \sqrt2\\
        -1 & \sqrt2 & 1\\
        -1 & -\sqrt2 & 1
        \end{pmatrix}.
        \]
        Now, calculating $A^TA$,
        \[
        A^TA = \frac14
        \begin{pmatrix}
        2 + 0 + 2 & -\sqrt2 + 0 - \sqrt2 & -\sqrt2 + 0 - \sqrt2\\
        -\sqrt2 + 0 - \sqrt2 & 1 + 2 + 1 & 1 - 2 + 1\\
        -\sqrt2 + 0 - \sqrt2 & 1 - 2 + 1 & 1 + 2 + 1
        \end{pmatrix} = 
        \begin{pmatrix}
        1 & 0 & 0\\
        0 & 1 & 0\\
        0 & 0 & 1
        \end{pmatrix} = I.
        \]
        Thus $A$ is orthogonal.

        \item [(b)]
        For an orthogonal matrix representing a rotation in $\mathbb{R}^3$, in 2D dimensions, the matrix can be represented by
        \[\begin{pmatrix}
        \cos\theta & -\sin\theta & 0\\
        \sin\theta & \cos\theta & 0\\
        0 & 0 & 1
        \end{pmatrix}.\]
        Thus Trace can be used to find the angle of rotation.
        For a rotation matrix in $\mathbb{R}^3$,
        \[\operatorname{tr}(A)=1+2\cos\theta.\]
        Here
        \[\operatorname{tr}(A)=\frac12(\sqrt2+\sqrt2+1)=\sqrt2+\frac12,\]
        so
        \[\cos\theta=\frac{\operatorname{tr}(A)-1}{2}=\frac{\left(\sqrt2+\frac12\right)-1}{2}=\frac{\sqrt2}{2}-\frac14.\]
        This gives
        \[\theta=\cos^{-1}\left(\frac{\sqrt2}{2}-\frac14\right)\approx \boxed{62.8^{\circ}}.\]

        The rotation axis is the eigenspace for eigenvalue $1$, i.e. solutions of
        \[(A-I)\mathbf{v}=0.\]
        A nonzero solution is
        \[\mathbf{v}=\begin{pmatrix}1+\sqrt2\\-(1+\sqrt2)\\1\end{pmatrix},\]
        so a unit vector along the axis is
        \[\hat{\mathbf{v}}=\frac{1}{\sqrt{(1+\sqrt2)^2+(1+\sqrt2)^2+1}}\begin{pmatrix}1+\sqrt2\\-(1+\sqrt2)\\1\end{pmatrix}=\boxed{\frac{1}{\sqrt{7+4\sqrt2}}\begin{pmatrix}1+\sqrt2\\-(1+\sqrt2)\\1\end{pmatrix}}.\]
    \end{enumerate}
\end{enumerate}

\newpage
\section*{Section B}

% =========================
% Q7
% =========================
\begin{mdframed}[style=exercise]
\noindent\textbf{7.}

\noindent\textbf{(a)} State de Moivre's theorem and show that
\begin{enumerate}
\item[(i)]
\[
\sum_{n=0}^{N-1}\cos n\theta
=
\frac{\sin(N\theta/2)}{\sin(\theta/2)}
\cos\frac{(N-1)\theta}{2},
\]
\item[(ii)]
\[
\cos5\theta
=
16\cos^5\theta-20\cos^3\theta+5\cos\theta,
\]
\end{enumerate}
where $n,N$ are integers and $\theta$ is a complex variable. \hfill [8]

\noindent\textbf{(b)} Find all the roots for the equation
\[
\left(\frac{z-1}{z+1}\right)^n=-1.
\tag{$\dagger$}
\]
Verify your solution for the special case $n=3$, by finding the roots of the resulting
third order equation.

\medskip
Use the general solution to $(\dagger)$ to calculate the product
\[
\prod_{r=1}^n
\cot\left(\frac{(2r+1)\pi}{2n}\right),
\]
where $r$ is an integer, for both odd and even values of $n$. \hfill [8]

\noindent\textbf{(c)} Show that if the complex numbers $z$ and $u$ satisfy the relation
\[
\left|\frac{z+u}{z+u^*}\right|=1,
\]
then either $u$ or $z$ must be real.

\medskip
\noindent [The $(*)$ stands for the complex conjugate.] \hfill [4]
\end{mdframed}

\begin{enumerate}
    \item [(a)] \textbf{de Moivre's theorem} states that for any real number $\theta$ and integer $n$,
    \[(\cos\theta + i\sin\theta)^n = \cos(n\theta) + i\sin(n\theta).\]
    \begin{enumerate}
        \item [(i)] Using the formula for the sum of a geometric series, we have
        \begin{align*}
            \sum_{n=0}^{N-1}\cos n\theta &= \sum_{n=0}^{N-1}\Re(\e^{in\theta}) \\
            &= \Re\left(\sum_{n=0}^{N-1}\e^{in\theta}\right) \\
            &= \Re\left(\frac{1-\e^{iN\theta}}{1-\e^{i\theta}}\right) \\
            &= \Re\left(\frac{\e^{i(N-1)\theta/2}}{\e^{i\theta/2}} \cdot \frac{\e^{iN\theta/2}-\e^{-iN\theta/2}}{\e^{i\theta/2}-\e^{-i\theta/2}}\right) \\
            &= \Re\left(\frac{\e^{i(N-1)\theta/2}}{\e^{i\theta/2}} \cdot \frac{\sin(N\theta/2)}{\sin(\theta/2)}\right) \\
            &= \frac{\sin(N\theta/2)}{\sin(\theta/2)} \cdot \Re(\e^{i(N-1)\theta/2}) \\
            &= \frac{\sin(N\theta/2)}{\sin(\theta/2)} \cos((N-1)\theta/2).
        \end{align*}
        \hfill Q.E.D.
        \item [(ii)] Using de Moivre's theorem, we have
        \begin{align*}
            \cos 5\theta + i\sin 5\theta &= (\cos\theta + i\sin\theta)^5 \\
            &= \cos^5\theta + 5i\cos^4\theta\sin\theta - 10\cos^3\theta\sin^2\theta - 10i\cos^2\theta\sin^3\theta + 5\cos\theta\sin^4\theta + i\sin^5\theta.
        \end{align*}
        Equating the real parts, we get
        \[\cos 5\theta = \cos^5\theta - 10\cos^3\theta\sin^2\theta + 5\cos\theta\sin^4\theta.\]
        Using the identity $\sin^2\theta = 1 - \cos^2\theta$, we can rewrite the equation as
        \begin{align*}
            \cos 5\theta &= \cos^5\theta - 10\cos^3\theta(1 - \cos^2\theta) + 5\cos\theta(1 - \cos^2\theta)^2 \\
            &= \cos^5\theta - 10\cos^3\theta + 10\cos^5\theta + 5\cos\theta - 10\cos^3\theta + 5\cos^5\theta \\
            &= 16\cos^5\theta - 20\cos^3\theta + 5\cos\theta.
        \end{align*}
        \hfill Q.E.D.
        \item [(b)] Rearranging the equation, we have
        \[\frac{z-1}{z+1} = \e^{i{(2k+1)\pi}/{n}}, \quad k=0,1,2,\ldots,n-1.\]
        Solving for $z$, we get
        \[\boxed{z = \frac{1 + \e^{i{(2k+1)\pi}/{n}}}{1 - \e^{i{(2k+1)\pi}/{n}}}}.\]
        For the special case $n=3$, we have
        \[z_k = \frac{1 + \e^{i{(2k+1)\pi}/{3}}}{1 - \e^{i{(2k+1)\pi}/{3}}}, \quad k=0,1,2.\]
        The three roots are
        \begin{align*}
            z_0 &= \frac{1 + \e^{i\pi/3}}{1 - \e^{i\pi/3}}, \\
            z_1 &= \frac{1 + \e^{i\pi}}{1 - \e^{i\pi}}, \\
            z_2 &= \frac{1 + \e^{i5\pi/3}}{1 - \e^{i5\pi/3}}.
        \end{align*}
        Simplifying these expressions, we get
        \begin{align*}
            z_0 &= \sqrt{3}i, \\
            z_1 &= 0, \\
            z_2 &= -\sqrt{3}i
            .
        \end{align*}
        These are the three roots of the equation for $n=3$.

        For the product, we have the equation
\[
\left(\frac{z-1}{z+1}\right)^n=-1.
\tag{$\dagger$}
\]
Let
\[
w=\frac{z-1}{z+1}.
\]
Then $w^n=-1=e^{i(2k+1)\pi}$, so
\[
w_k=e^{i(2k+1)\pi/n},\qquad k=0,1,\dots,n-1.
\]
Solving $z-1=w(z+1)$ gives
\[
z=\frac{1+w}{1-w}.
\]
With $w=e^{i\phi}$ we have the standard identity
\[
\frac{1+e^{i\phi}}{1-e^{i\phi}}=-i\cot\!\left(\frac{\phi}{2}\right),
\]
hence
\[
z_k=-i\cot\!\left(\frac{(2k+1)\pi}{2n}\right).
\]
Since the set of angles
\[
\left\{\frac{(2r+1)\pi}{2n}: r=1,\dots,n\right\}
\]
is the same as
\[
\left\{\frac{(2k+1)\pi}{2n}: k=0,\dots,n-1\right\}
\]
(up to a permutation, using $\cot(\pi+\alpha)=\cot\alpha$),
the required product is
\[
P:=\prod_{r=1}^n \cot\!\left(\frac{(2r+1)\pi}{2n}\right)
=\prod_{k=0}^{n-1}\cot\!\left(\frac{(2k+1)\pi}{2n}\right).
\]
From $z_k=-i\cot(\cdot)$ we get $\cot(\cdot)=iz_k$, so
\[
P=i^n\prod_{k=0}^{n-1} z_k.
\]

Now $(\dagger)$ is equivalent to
\[
(z-1)^n=-(z+1)^n
\quad\Longleftrightarrow\quad
(z-1)^n+(z+1)^n=0,
\]
Expanding the left-hand side using the binomial theorem, we see that only the even powers of $z$ survive, with coefficients
\[\binom{n}{0}+\binom{n}{2}+\binom{n}{4}+\cdots = 2^{n-1},\]
so the polynomial is of degree $n$ with leading coefficient $2^{n-1}$. The $z_k$ are the $n$ roots of this polynomial, 
\[F(z):=(z-1)^n+(z+1)^n=0.
\]
We know that
\[2(z-z_0)(z-z_1)\cdots(z-z_{n-1})=F(z)/2^{n-1}.\]
Setting $z=0$ gives
\[\prod_{k=0}^{n-1} z_k = (-1)^n \frac{F(0)}{2^n}=(-1)^n\frac{(-1)^n+1}{2}.
\]
Therefore
\[
P=i^n(-1)^n\frac{(-1)^n+1}{2}.
\]
If $n$ is odd, then $(-1)^n+1=0$ and so $P=0$.
If $n$ is even, then $(-1)^n+1=2$ and $\prod z_k=1$, so
\[
P=i^n=i^{2(n/2)}=(-1)^{n/2}.
\]

\[
\boxed{
\prod_{r=1}^n \cot\!\left(\frac{(2r+1)\pi}{2n}\right)=
\begin{cases}
0, & n\ \text{odd},\\[4pt]
(-1)^{n/2}, & n\ \text{even}.
\end{cases}}
\]


    \item [(c)] Given that
\[\left|\frac{z+u}{z+u^*}\right|=1,\]
we have
\[\left|\frac{z+u}{z+u^*}\right|^2 = \frac{(z+u)(z^*+u^*)}{(z+u^*)(z^*+u)} = 1.\]
Expanding both sides, we get
\begin{align*}
(z+u)(z^*+u^*) &= (z+u^*)(z^*+u)\\
zz^* + zu^* + uz^* + uu^* &= zz^* + zu + u^*z^* + u^*u\\
zu^* + uz^* &= zu + u^*z^*\\
zu^* - zu &= u^*z^* - uz^*\\
z(u^* - u) &= z^*(u^* - u)\\
(z - z^*)(u^* - u) &= 0.
\end{align*}
Thus, either $z - z^* = 0$ or $u^* - u = 0$, which means either $z\in \mathbb{R}$ or $u \in \mathbb{R}$ . \hfill Q.E.D.


    \end{enumerate}
\end{enumerate}













% =========================
% Q8
% =========================
\begin{mdframed}[style=exercise]
\noindent\textbf{8.}
The differential equation for the displacement $y(t)$ of a particle executing forced
and damped harmonic oscillations with damping factor $\gamma$ and natural frequency
$\omega_0$ may be written as
\[
\frac{\mathrm{d}^2y}{\mathrm{d}t^2}+2\gamma\frac{\mathrm{d}y}{\mathrm{d}t}+\omega_0^2 y
=F\cos\omega t,
\]
where $F$ and $\omega$ are the amplitude and frequency of the driving force respectively.

\noindent\textbf{(a)} Assuming that $F=0$, find the displacements $y(t)$ of the particle for the
cases $\omega_0<\gamma$ and $\omega_0=\gamma$. Sketch and compare the two displacements.
\hfill [6]

\noindent\textbf{(b)} Assume now that $F=F_0\neq0$ and $\gamma<\omega_0$. Explain what is meant
by a steady state solution of the differential equation and find an expression for the steady
state amplitude and phase of the displacement.

\medskip
For a given value of the natural frequency $\omega_0$, which value of the driving force
frequency $\omega$ maximises the displacement? For what value of $\omega$ is the velocity
a maximum? \hfill [6]

\noindent\textbf{(c)} Explain what is meant by the width of the oscillator resonance. Calculate the
width of the resonance for the case $\gamma\ll\omega_0$. \hfill [4]
\end{mdframed}


\begin{enumerate}
    \item [(a)] For $F=0$, the differential equation becomes
    \[\frac{\mathrm{d}^2y}{\mathrm{d}t^2}+2\gamma\frac{\mathrm{d}y}{\mathrm{d}t}+\omega_0^2 y = 0.\]
    The characteristic equation is
    \[r^2 + 2\gamma r + \omega_0^2 = 0.\]
    The roots are
    \[r = -\gamma \pm \sqrt{\gamma^2 - \omega_0^2}.\]
    \begin{itemize}
        \item For $\omega_0 < \gamma$, the roots are real and distinct. The general solution is
        \[y(t) = \e^{-\gamma t}(C_1 \e^{\sqrt{\gamma^2 - \omega_0^2}t} + C_2 \e^{-\sqrt{\gamma^2 - \omega_0^2}t}),\]
        where $C_1$ and $C_2$ are constants determined by initial conditions.

        \item For $\omega_0 = \gamma$, the roots are real and equal. The general solution is
        \[y(t) = (C_1 + C_2 t)\e^{-\gamma t},\]
        where $C_1$ and $C_2$ are constants determined by initial conditions.
    \end{itemize}
    The two displacements can be sketched as follows:

    \begin{figure}[htbp]
        \centering
        \includegraphics[width=0.9\textwidth]{1.png}
        \caption{Displacements for $\omega_0 < \gamma$}
    \end{figure}

    \begin{figure}[htbp]
        \centering
        \includegraphics[width=0.9\textwidth]{2.png}
        \caption{Displacements for $\omega_0 = \gamma$}
    \end{figure}

    \item [(b)] A steady state solution is a particular solution of the differential equation that represents the long-term behavior of the system after transient effects have died out. It is typically periodic and has the same frequency as the driving force.
    
    To find the steady state solution, we assume a solution of the form
    \[y(t) = A\cos(\omega t - \delta),\]
    where $A$ is the amplitude and $\delta$ is the phase shift. Substituting this into the differential equation, we get
    \[-A\omega^2\cos(\omega t - \delta) - 2\gamma A\omega\sin(\omega t - \delta) + \omega_0^2 A\cos(\omega t - \delta) = F_0\cos(\omega t).\]
    Equating coefficients of $\cos(\omega t)$ and $\sin(\omega t)$, we have
    \[A(\omega_0^2 - \omega^2) = F_0\cos\delta,\]
    \[2\gamma A\omega = F_0\sin\delta.\]
    Solving for $A$ and $\delta$, we get
    \[A = \frac{F_0}{\sqrt{(\omega_0^2 - \omega^2)^2 + (2\gamma\omega)^2}},\]
    \[\tan\delta = \frac{2\gamma\omega}{\omega_0^2 - \omega^2}.\]
    To maximize the displacement, we differentiate $A$ with respect to $\omega$ and set it to zero. The maximum occurs when $(\omega_0^2 - \omega^2)^2 + (2\gamma\omega)^2$ is minimized. This happens when $\boxed{\omega = \sqrt{\omega_0^2 - 2\gamma^2}}$, provided $\omega_0 > \gamma$.
    The velocity is given by
    \[v = \omega A= \frac{\omega F_0}{\sqrt{(\omega_0^2 - \omega^2)^2 + (2\gamma\omega)^2}}.\]
    To maximize the velocity, we differentiate $v$ with respect to $\omega$ and set it to zero. The maximum occurs when minimizing 
    \[\left(\left(\frac{\omega_0}{\omega}\right)^2-1\right)^2+4\gamma^2.\]
    This happens when $\boxed{\omega = \omega_0}$.


    \item [(c)] The width of the oscillator resonance, also known as the bandwidth, is defined as the range of frequencies over which the amplitude of the oscillation is greater than or equal to $\displaystyle\frac{A_{\text{max}}}{2}$, where $A_{\text{max}}$ is the maximum amplitude at resonance.
    For $\gamma \ll \omega_0$, the resonance occurs at $\omega \approx \omega_0$. The amplitude at resonance is
    \[A_{\text{max}} = \frac{F_0}{2\gamma\omega_0}.\]
    To find the frequencies at which the amplitude is half of $A_{\text{max}}$, we set
    \[\frac{F_0}{\sqrt{(\omega_0^2 - \omega^2)^2 + (2\gamma\omega)^2}} = \frac{A_{\text{max}}}{2} = \frac{F_0}{4\gamma\omega_0}.\]
    Squaring both sides and simplifying, we get
    \[(\omega_0^2 - \omega^2)^2 + (2\gamma\omega)^2 = (4\gamma\omega_0)^2.\]
    Solving this quadratic equation for $\omega$, we find the two frequencies $\omega_1$ and $\omega_2$ at which the amplitude is half of $A_{\text{max}}$. The width of the resonance is then given by
    \[\Delta \omega = \omega_2 - \omega_1 \approx\boxed{ 2\gamma}.\]



\end{enumerate}









% =========================
% Q9
% =========================
\begin{mdframed}[style=exercise]
\noindent\textbf{9.}
The matrix
\[
A=
\begin{pmatrix}
1 & -1 & \alpha\\
\alpha-3 & 0 & 1\\
2 & -1 & \alpha+1
\end{pmatrix}
\]
defines the linear map $f:\mathbb{R}^3\to\mathbb{R}^3$ by $f(x)=Ax$, where $x\in\mathbb{R}^3$
and $\alpha\in\mathbb{R}$.

\medskip
Let $A_4$ be equal to the matrix $A$ for $\alpha=4$.

\noindent\textbf{(a)} Find a basis for $\ker(f)$ and show that the geometry of the kernel is a straight
line in $\mathbb{R}^3$. Write the equation of the line in vector form. \hfill [4]

\noindent\textbf{(b)} By choosing $(\mathbf{e}_1,\mathbf{e}_2,\mathbf{e}_3)$ as a basis of $\mathbb{R}^3$
with $\mathbf{e}_1\in\ker(f)$, show that the geometry of the image of $f$ is a plane and find
the direction of the normal to this plane. \hfill [5]

\noindent Now let $A_0$ be equal to $A$ for $\alpha=0$.

\noindent\textbf{(c)} Assuming that the matrix $A_0$ was calculated with respect to the basis
\[
\mathbf{u}_1=(1,0,0)^T,\quad
\mathbf{u}_2=(0,1,0)^T,\quad
\mathbf{u}_3=(0,0,1)^T,
\]
express the map $f(x)$ in terms of $\mathbf{u}_1,\mathbf{u}_2,\mathbf{u}_3$ and the coordinates
of the vector $x=(x_1,x_2,x_3)^T$. \hfill [3]

\noindent\textbf{(d)} Assume now that the matrix $A'_0$ of the map $f$ is calculated with respect to
the basis
\[
\mathbf{w}_1=(1,1,0)^T,\quad
\mathbf{w}_2=(1,0,1)^T,\quad
\mathbf{w}_3=(0,1,1)^T.
\]
Calculate the matrix $A'_0$ from the relation
\[
A'_0=CA_0C^{-1}
\]
by a suitable choice of the matrix $C$. \hfill [8]
\end{mdframed}

% =========================
% Q10
% =========================
\begin{mdframed}[style=exercise]
\noindent\textbf{10.}

\noindent\textbf{(a)} The components of a vector $\mathbf{a}$ in the $(x,y)$ plane of a Cartesian
coordinate system $(x,y,z)$ are given by $(a_x,a_y)$. Assume now that the $(x,y)$ axes are
rotated about the $z$ axis by an angle $\theta$, anticlockwise, so that the components of
$\mathbf{a}$ with respect to the rotated coordinate system $(x',y',z)$ are given by $(a'_x,a'_y)$.

\medskip
Calculate the elements of the matrix $R$ that relates the vector $(a'_x,a'_y)^T$ to the vector
$(a_x,a_y)^T$ and show that $R$ is a rotation matrix. Find the eigenvectors of the matrix $R$.
\hfill [7]

\noindent\textbf{(b)} The equation of a conical section in the $(x,y)$ coordinate system in $\mathbb{R}^2$
is given by
\[
f(x,y)=x^2+6xy+y^2=4.
\tag{$*$}
\]
Write down the above equation in the matrix form
\[
\mathbf{x}^TM\mathbf{x}=4
\]
where $M$ is a symmetric matrix and $\mathbf{x}=(x,y)^T$ is a coordinate vector in $\mathbb{R}^2$.
Use matrix diagonalisation to show that the curve in $(*)$ represents a hyperbola. Find the
elements of the unitary matrix that diagonalises the matrix $M$.

\medskip
Sketch this curve showing the asymptotes and the points of intersection with the $(x,y)$ axes.
\hfill [8]
\end{mdframed}




\end{document}